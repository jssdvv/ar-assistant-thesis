\section{Objetivos}

    \subsection{Objetivo general}

        Diseñar un aplicativo móvil de guías visuales instruccionales, con el uso de la realidad aumentada, como asistente en la capacitación de técnicos y el desarrollo en tareas de mantenimiento preventivo.
    
    \subsection{Objetivos específicos}

        \begin{itemize}
            
            \item Definir los requerimientos y restricciones asociados al desarrollo de un aplicativo de realidad aumentada enfocado en el mantenimiento preventivo, basándose en los conocimientos descritos por manuales y normas, para establecer las características funcionales y no funcionales de la aplicación.
        
            \item Desarrollar el software de realidad aumentada del aplicativo, en base a sus características de funcionamiento y el uso marcadores de posicionamiento de objetos, que faciliten la superposición de elementos digitales en tiempo real mediante el uso de la cámara del dispositivo.
        
            \item Evaluar el uso del aplicativo en un compresor de aire de 3HP ubicado en el laboratorio de potencia fluida de la escuela de ingeniería mecánica de la Universidad Industrial de Santander, verificando mediante la visualización interactiva de guías y documentación, su desempeño como herramienta de apoyo en el aprendizaje de actividades de mantenimiento preventivo.

        \end{itemize}
