\section{Introducción}

    En la actualidad las industrias se han encaminado hacia un entorno automatizado donde su objetivo principal tal como lo menciona \citet{MuñozMacias}, consiste en mejorar la eficiencia y confiabilidad de sus operaciones. Uno de los departamentos más relevantes en las industrias es el de mantenimiento, donde este ha decidido evolucionar desde los métodos tradicionales y menos proactivos a unos más enfocados en la planificación y prevención asistidos por las nuevas tecnologías. Sin embargo, tal como lo menciona el Centro de Desarrollo Humano de \citet{FundacionChile}, “Esto generalmente se logra mediante una combinación personalizada de software, prácticas y personal que se enfoca en lograr dichos objetivos”. 

    Uno de los enfoques de numerosos grupos de investigación, tal como lo menciona \citet{MuñozMacias}, es el entrenamiento de operadores en la realización óptima de tareas de mantenimiento y procesos de ensamblado. Una de las tecnologías con mayor potencial al mejoramiento de dichas tareas es la denominada realidad aumentada (RA) la cual, tal como lo describe \citet{ArenasFabio}, consiste en crear un enlace directo o provocado por medio de la interacción del usuario entre el mundo real y la información generada por un dispositivo electrónico. Esta adopción tecnológica ha abierto consigo un nuevo concepto conocido como “Realidad Aumentada Industrial” (IAR) la cual \citet{BondinAndrea} describen como un apoyo hacia procesos industriales gracias a su implementación en sistemas de manufactura. 

    La aplicación de la realidad aumentada en el ámbito del mantenimiento industrial trae consigo beneficios tales como “…la reducción de tiempos de formación, la facilitación de tareas de mantenimiento, el acceso inmediato a información crítica y la disminución de riesgos laborales” \citep{MuñozMacias}. Donde, es esencial que el desarrollo de una herramienta de realidad aumentada cumpla con este enfoque de practicidad para complementar el entorno laboral de los usuarios que hagan uso de esta y a su vez facilitando sus procesos de formación mediante un aprendizaje interactivo.
